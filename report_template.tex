% report_template.tex - template minimal for the project
\documentclass[a4paper,11pt]{article}
\usepackage[utf8]{inputenc}
\usepackage{graphicx}
\usepackage{booktabs}
\usepackage{hyperref}
\usepackage{amsmath}
\usepackage{caption}
\usepackage{float}
\title{Reinforcement Learning on Taxi-v3: Q-learning, DQN, Double DQN}
\author{Nome Cognome - Matricola}
\date{\today}

\begin{document}
\maketitle
\begin{abstract}
Breve sommario degli obiettivi, metodi e risultati principali.
\end{abstract}

\section{Introduzione}
Motivazioni e obiettivi.

\section{Background}
Brevi richiami su MDP, Q-learning, Bellman, DQN, Double DQN.

\section{Ambiente}
Descrizione di Taxi-v3: stati, azioni, reward.

\section{Metodi}
Spiegazione matematica e pseudocodice di:
\begin{itemize}
  \item Q-learning tabellare
  \item DQN (single network)
  \item Double DQN
\end{itemize}

\section{Setup sperimentale}
Hardware, versioni librerie (includere requirements.txt), iperparametri (tabella).

\section{Risultati}
\begin{figure}[H]
\centering
\includegraphics[width=0.9\textwidth]{results/comparison.png}
\caption{Confronto media $\pm$ std tra metodi (più seed).}
\end{figure}

Aggiungere tabelle con metriche: avg reward, std, episodi di convergenza, tempo.

\section{Discussione}
Analisi critica, cause di comportamenti osservati, limiti.

\section{Conclusioni e lavori futuri}
Riepilogo e suggerimenti (Prioritized Replay, Dueling, etc.)

\appendix
\section{Appendice - Hyperparameters}
Inserire file params.json o screenshot.

\end{document}
